\documentclass[a4paper,10pt]{article}
\usepackage[english, brazil]{babel}
\usepackage[utf8]{inputenc}
\usepackage{amsmath}
\usepackage{graphicx}
\usepackage{float}
\usepackage[active]{srcltx}
\usepackage[all]{xy}
\usepackage{color}
\usepackage{parcolumns}
\usepackage{lscape}
\usepackage{psfrag}
\usepackage{setspace}
%opening
\title{Implementação do método FDTD para solução das Equações de Maxwell com UPML com condição de contorno}
\author{Dr. Maicon Saul Faria}

\begin{document}

\maketitle

\begin{abstract}
Este documento serve como referencia das equações e dos procedimentos usados na implementação de Uniaxial Perfect Matched Layer UPML nos algoritmos FDTD para paralelismo em GPU.
\end{abstract}

\section{Introdução}
A técnica UPML consiste em considerar um material anisotrópico como contorno do domínio computacional e escolher as componentes adequadas dos tensores permissividade elétrica  e permeabilidade magnética. Desta forma obtemos contornos que não refletem ondas planas para qualquer ângulo incidente e absorvem estas ondas fazendo com que os campos que decaiam exponencialmente com  a distância. Os detalhes desta técnica podem ser encontrados no Taflove 2a edição no capítulo 7.
Inicialmente vamos considerar as equações de Faraday (\ref{Faraday}) e Ampere (\ref{Ampere}) onde foi aplicada a transformadas de Fourier temporal nos campos:
\begin{equation}
 \nabla \times \mathbf{\hat{H}} = j \omega \epsilon \, \mathbf{\bar{S}}\, \mathbf{\hat{E}}\,, \label{Ampere}
\end{equation} 

\begin{equation}
\nabla \times \mathbf{\hat{E}} = j \omega \mu \, \mathbf{\bar{S}}\, \mathbf{\hat{H}}\,,\label{Faraday}
\end{equation} 
onde $\mathbf{\hat{E}}$ e $\mathbf{\hat{H}}$ são as transformadas de Fourier do campo elétrico e magnético respectivamente, $\epsilon$ é a permissividade elétrica e $\mu$ a permeabilidade magnética do meio, $\mathbf{\bar{S}}$ é o tensor que proporciona a anisotropia das propriedades eletromagnéticas.

O tensor $\mathbf{\bar{S}}$ escolhido para minimizar a reflexão para ondas incidentes é dado pela equação (\ref{tensor}),

\begin{equation}
  \mathbf{\bar{S}} = \begin{Vmatrix}
 \frac{s_y s_z}{s_x} &  &  \\
  & \frac{s_x s_z}{s_y} &  \\
  &  & \frac{s_x s_y}{s_z}
\end{Vmatrix}\,, \label{tensor}
\end{equation}
onde, 

\begin{equation}
 s_l=\kappa_l + \frac{\sigma_l}{j\omega \epsilon}\,,  \label{sigma}
\end{equation}
com $l=x,y,z$. O valores dos parâmetros $\kappa_l$ e $\sigma_l$ serão descritos mais adiante, $\omega$ é o termo de frequência da transformada de Fourier.

As definições das relações constitutivas, dadas pelas equações (\ref{cons1}-\ref{cons6}), são escolhidas de forma a evitar constantes dependentes de $\omega$ e assim evitar a necessidade de convoluções.

\begin{equation}
\mathbf{\hat{D}_x}=\dfrac{\epsilon s_z}{s_x}\mathbf{\hat{E}_x}\,, \label{cons1}
\end{equation}
\begin{equation}
\mathbf{\hat{D}_y}=\dfrac{\epsilon s_x}{s_y}\mathbf{\hat{E}_y}\,,\label{cons2}
\end{equation}
\begin{equation}
\mathbf{\hat{D}_z}=\dfrac{\epsilon s_y}{z_x} \mathbf{\hat{E}_z}\,,\label{cons3}
\end{equation}
\begin{equation}
\mathbf{\hat{B}_x}=\dfrac{\mu s_z}{s_x}\mathbf{\hat{H}_x}\,,\label{cons4}
\end{equation}
\begin{equation}
\mathbf{\hat{B}_y}=\dfrac{\mu s_x}{s_y}\mathbf{\hat{H}_y}\,,\label{cons5}
\end{equation}
\begin{equation}
\mathbf{\hat{B}_z}=\dfrac{\mu s_y}{s_z}\mathbf{\hat{H}_z}\,.\label{cons6}
\end{equation}

Substituindo as equações (\ref{cons1}-\ref{cons6}) em  (\ref{Faraday}) e (\ref{Ampere}) obtemos,

\begin{equation}
  \begin{vmatrix}
 \dfrac{\partial \hat{H}_z}{\partial y} - \dfrac{\partial \hat{H}_y}{\partial z}  \\
\\
 \dfrac{\partial \hat{H}_x}{\partial z} - \dfrac{\partial \hat{H}_z}{\partial x}  \\
\\ 
\dfrac{\partial \hat{H}_y}{\partial x} - \dfrac{\partial \hat{H}_x}{\partial y}  \\
  \end{vmatrix} 
= i \omega \begin{Vmatrix}
 s_y &  &  \\ \\
  & s_z &  \\ \\
  &  & s_x 
\end{Vmatrix} 
 \begin{vmatrix}
 \hat{D}_x \\ \\
 \hat{D}_y \\ \\
 \hat{D}_z \\ \\
  \end{vmatrix}  \label{tensor-Ampere}
\end{equation}

\begin{equation}
 \begin{vmatrix}
 \dfrac{\partial \hat{E}_z}{\partial y} - \dfrac{\partial \hat{E}_y}{\partial z}  \\
\\
 \dfrac{\partial \hat{E}_x}{\partial z} - \dfrac{\partial \hat{E}_z}{\partial x}  \\
\\ 
\dfrac{\partial \hat{E}_y}{\partial x} - \dfrac{\partial \hat{E}_x}{\partial y}  \\
  \end{vmatrix} 
= -j \omega \begin{Vmatrix}
 s_y &  &  \\ \\
  & s_z &  \\ \\
  &  & s_x 
\end{Vmatrix} 
 \begin{vmatrix}
 \hat{H}_x \\ \\
 \hat{H}_y \\ \\
 \hat{H}_z \\ \\
  \end{vmatrix}  \label{tensor-Faraday}
\end{equation}

Usando a equação (\ref{sigma}) em  e (\ref{tensor-Ampere}), (\ref{tensor-Faraday}) e tomando a transformada inversa de Fourier,

\begin{equation}
\begin{vmatrix}
 \dfrac{\partial H_z}{\partial y} - \dfrac{\partial H_y}{\partial z}  \\
\\
 \dfrac{\partial H_x}{\partial z} - \dfrac{\partial H_z}{\partial x}  \\
\\ 
\dfrac{\partial H_y}{\partial x} - \dfrac{\partial H_x}{\partial y}  \\
  \end{vmatrix} 
= \dfrac{\partial}{\partial t} \begin{Vmatrix}
 \kappa_y &  &  \\ \\
  & \kappa_z &  \\ \\
  &  & \kappa_x 
\end{Vmatrix} 
 \begin{vmatrix}
 D_x \\ \\
 D_y \\ \\
 D_z 
 \end{vmatrix} +  
\frac{1}{\epsilon}
\begin{Vmatrix}
 \sigma_y &  &  \\ \\
  & \sigma_z &  \\ \\
  &  & \sigma_x 
\end{Vmatrix} 
 \begin{vmatrix}
 D_x \\ \\
 D_y \\ \\
 D_z 
  \end{vmatrix} \label{base-Ampere}
\end{equation}

\begin{equation}
\begin{vmatrix}
 \dfrac{\partial E_z}{\partial y} - \dfrac{\partial E_y}{\partial z}  \\
\\
 \dfrac{\partial E_x}{\partial z} - \dfrac{\partial E_z}{\partial x}  \\
\\ 
\dfrac{\partial E_y}{\partial x} - \dfrac{\partial E_x}{\partial y}  \\
  \end{vmatrix} 
= -\dfrac{\partial}{\partial t} \begin{Vmatrix}
 \kappa_y &  &  \\ \\
  & \kappa_z &  \\ \\
  &  & \kappa_x 
\end{Vmatrix} 
 \begin{vmatrix}
 B_x \\ \\
 B_y \\ \\
 B_z 
 \end{vmatrix}  
-\frac{1}{\epsilon}
\begin{Vmatrix}
 \sigma_y &  &  \\ \\
  & \sigma_z &  \\ \\
  &  & \sigma_x 
\end{Vmatrix} 
 \begin{vmatrix}
 B_x \\ \\
 B_y \\ \\
 B_z 
  \end{vmatrix} \label{base-Faraday}
\end{equation}

\section{Caso 2-d TEz}

As derivadas em  se anulam. As componentes das equações (\ref{base-Ampere}) e (\ref{base-Faraday}) necessárias são:

\begin{equation}
\dfrac{\partial H_z}{\partial y}=\kappa_y\dfrac{\partial D_x}{\partial t}+ \frac{\sigma_y}{\epsilon}D_x
\end{equation}

\begin{equation}
\dfrac{\partial H_z}{\partial x}=-\kappa_z\dfrac{\partial D_y}{\partial t}- \frac{\sigma_z}{\epsilon}D_y
\end{equation}

\begin{equation}
\dfrac{\partial E_y}{\partial x}-\dfrac{\partial E_x}{\partial y}=-\kappa_x\dfrac{\partial B_z}{\partial t}- \frac{\sigma_x}{\epsilon}B_z
\end{equation}

As equações são discretizadas segundo o esquema de Yee:

\begin{align}
\frac{H_z |_{i,j+1}^{n}-H_z |_{i,j}^{n}}{\Delta y}
= \kappa_y \frac{D_x |_{i,j+1/2}^{n+1/2}-D_x |_{i,j+1/2}^{n-1/2}}{\Delta t} \nonumber
\\+ \frac{\sigma_y}{\epsilon}\frac{D_x |_{i,j+1/2}^{n+1/2}+D_x |_{i,j+1/2}^{n-1/2}}{2}\,, \label{discr1}
\end{align}

\begin{align}
\frac{H_z |_{i+1,j}^{n}-H_z |_{i,j}^{n}}{\Delta x}
= - \kappa_z \frac{D_y |_{i+1/2,j}^{n+1/2}-D_y |_{i+1/2,j}^{n-1/2}}{\Delta t} \nonumber
\\ - \frac{\sigma_z}{\epsilon}\frac{D_y |_{i+1/2,j}^{n+1/2}+D_y |_{i+1/2,j}^{n-1/2}}{2}\,, \label{discr2}
\end{align}

\begin{align}
\frac{E_y |_{i+1/2,j}^{n+1/2}-E_y |_{i-1/2,j}^{n+1/2}}{\Delta x}
-\frac{E_x |_{i,j+1/2}^{n+1/2}-E_x |_{i,j-1/2}^{n+1/2}}{\Delta y} \nonumber
\\= \kappa_x \frac{ B_z |_{i,j}^{n+1}-B_z |_{i,j}^{n}}{\Delta t}
+ \frac{\sigma_x}{\epsilon}\frac{B_z |_{i,j}^{n+1}+B_z |_{i,j}^{n}}{2}\,. \label{discr3}
\end{align}


Devemos discretizar também as relações constitutivas (\ref{cons1}), (\ref{cons2}) e (\ref{cons6}), substituindo (\ref{sigma}) nessas relações e tomando a transformada de Fourier inversa, obtemos:

\begin{equation}
\kappa_x \dfrac{\partial D_x}{\partial t} + \frac{\sigma_x}{\epsilon}D_x=
\epsilon \left( \kappa_z \dfrac{\partial E_x}{\partial t} + \frac{\sigma_z}{\epsilon}E_x\right)\,,
\end{equation}

\begin{equation}
\kappa_y \dfrac{\partial D_y}{\partial t} + \frac{\sigma_y}{\epsilon}D_y=
\epsilon \left( \kappa_x \dfrac{\partial E_y}{\partial t} + \frac{\sigma_x}{\epsilon}E_y\right)\,,
\end{equation}

\begin{equation}
\kappa_z \dfrac{\partial B_z}{\partial t} + \frac{\sigma_z}{\epsilon}H_z=
\mu \left( \kappa_y \dfrac{\partial H_z}{\partial t} + \frac{\sigma_y}{\epsilon}H_z\right)\,.
\end{equation}

Discretizando usando novamente o esquema de Yee:

\begin{align}
\kappa_x \frac{ D_x |_{i,j+1/2}^{n+1/2}-D_x |_{i,j+1/2}^{n-1/2}}{\Delta t}
+ \frac{\sigma_x}{\epsilon}\frac{D_x |_{i,j+1/2}^{n+1/2}+D_x |_{i,j+1/2}^{n-1/2}}{2}= \nonumber
\\ \epsilon \left( \kappa_z \frac{ E_x |_{i,j+1/2}^{n+1/2}-E_x |_{i,j+1/2}^{n-1/2}}{\Delta t}
+ \frac{\sigma_z}{\epsilon}\frac{E_x |_{i,j+1/2}^{n+1/2}+E_x |_{i,j+1/2}^{n-1/2}}{2}\right)\,, \label{discr4}
\end{align}

\begin{align}
\kappa_y \frac{ D_y |_{i+1/2,j}^{n+1/2}-D_y |_{i+1/2,j}^{n-1/2}}{\Delta t}
+ \frac{\sigma_y}{\epsilon}\frac{D_y |_{i+1/2,j}^{n+1/2}+D_y |_{i+1/2,j}^{n-1/2}}{2}= \nonumber
\\ \epsilon \left( \kappa_x \frac{ E_y |_{i+1/2,j}^{n+1/2}-E_y |_{i+1/2,j}^{n-1/2}}{\Delta t}
+ \frac{\sigma_x}{\epsilon}\frac{E_y |_{i+1/2,j}^{n+1/2}+E_y |_{i+1/2,j}^{n-1/2}}{2}\right)\,, \label{discr5}
\end{align}

\begin{align}
\kappa_z \frac{ B_z |_{i,j}^{n+1}-B_z |_{i,j}^{n}}{\Delta t}
+ \frac{\sigma_z}{\epsilon}\frac{B_z |_{i,j}^{n+1}+B_z |_{i,j}^{n}}{2}\nonumber
\\ =\mu \left( \kappa_y \frac{ H_z |_{i,j}^{n+1}-H_z |_{i,j}^{n}}{\Delta t}
+ \frac{\sigma_y}{\epsilon}\frac{H_z |_{i,j}^{n+1}+H_z |_{i,j}^{n}}{2}\right)\,. \label{discr6}
\end{align}

Isolando adequadamente os termos das equações (\ref{discr1}-\ref{discr3}) e (\ref{discr4}-\ref{discr6}),

\begin{equation}
 \left( \frac{\kappa_y}{\Delta t} + \frac{\sigma_y}{2 \epsilon} \right) D_x |_{i,j+1/2}^{n+1/2} = 
\left( \frac{\kappa_y}{\Delta t} - \frac{\sigma_y}{2 \epsilon} \right) D_x |_{i,j+1/2}^{n-1/2}+
\frac{H_z |_{i,j+1}^{n}-H_z |_{i,j}^{n}}{\Delta y}\,
\end{equation}

\begin{equation}
 \left( \frac{\kappa_z}{\Delta t} + \frac{\sigma_z}{2 \epsilon} \right) D_y |_{i+1/2,j}^{n+1/2} = 
\left( \frac{\kappa_z}{\Delta t} - \frac{\sigma_z}{2 \epsilon} \right) D_y |_{i+1/2,j}^{n-1/2} -
\frac{H_z |_{i+1,j}^{n}-H_z |_{i,j}^{n}}{\Delta x}\,
\end{equation}

\begin{equation}
\left( \frac{\kappa_x}{\Delta t} + \frac{\sigma_x}{2 \epsilon} \right) B_z |_{i,j}^{n+1} = 
\left( \frac{\kappa_x}{\Delta t} - \frac{\sigma_x}{2 \epsilon} \right) B_z |_{i,j}^{n} +
\frac{E_y |_{i+1/2,j}^{n+1/2}-E_y |_{i-1/2,j}^{n+1/2}}{\Delta x}
-\frac{E_x |_{i,j+1/2}^{n+1/2}-E_x |_{i,j-1/2}^{n+1/2}}{\Delta y}\,
\end{equation}

\begin{equation}
\left( \frac{\kappa_z}{\Delta t} + \frac{\sigma_z}{2 \epsilon} \right) E_x |_{i,j+1/2}^{n+1/2} =
\left( \frac{\kappa_z}{\Delta t} - \frac{\sigma_z}{2 \epsilon} \right) E_x |_{i,j+1/2}^{n-1/2} +
\frac{1}{\epsilon} \left( \frac{\kappa_x}{\Delta t} + \frac{\sigma_x}{2 \epsilon} \right)  D_x |_{i,j+1/2}^{n+1/2} +
\frac{1}{\epsilon} \left( \frac{\kappa_x}{\Delta t} - \frac{\sigma_x}{2 \epsilon} \right)  D_x |_{i,j+1/2}^{n-1/2}\,
\end{equation}

\begin{equation}
\left( \frac{\kappa_x}{\Delta t} + \frac{\sigma_x}{2 \epsilon} \right) E_y |_{i+1/2,j}^{n+1/2} =
\left( \frac{\kappa_x}{\Delta t} - \frac{\sigma_x}{2 \epsilon} \right) E_y |_{i+1/2,j}^{n-1/2} +
\frac{1}{\epsilon} \left( \frac{\kappa_y}{\Delta t} + \frac{\sigma_y}{2 \epsilon} \right)  D_y |_{i+1/2,j}^{n+1/2} -
\frac{1}{\epsilon} \left( \frac{\kappa_y}{\Delta t} - \frac{\sigma_y}{2 \epsilon} \right)  D_y |_{i+1/2,j}^{n-1/2}\,
\end{equation}

\begin{equation}
\left( \frac{\kappa_y}{\Delta t} + \frac{\sigma_y}{2 \epsilon} \right) H_z |_{i,j}^{n+1} = 
\left( \frac{\kappa_y}{\Delta t} - \frac{\sigma_y}{2 \epsilon} \right) H_z |_{i,j}^{n} +
\frac{1}{\mu} \left( \frac{\kappa_z}{\Delta t} + \frac{\sigma_z}{2 \epsilon} \right) B_z |_{i,j}^{n+1} -
\frac{1}{\mu} \left( \frac{\kappa_z}{\Delta t} - \frac{\sigma_z}{2 \epsilon} \right) B_z |_{i,j}^{n}\,.
\end{equation}

\section{Caso 3-d}

Os elementos da equação \ref{tensor-Ampere} \ref{tensor-Faraday} são

\begin{equation}
 \dfrac{\partial H_z}{\partial y} - \dfrac{\partial H_y}{\partial z} = \kappa_y \dfrac{\partial D_x}{\partial t} + \frac{\sigma_y}{\epsilon} \partial D_x \label{pre-disc1}
\end{equation}

\begin{equation}
 \dfrac{\partial H_x}{\partial z} - \dfrac{\partial H_z}{\partial x} = \kappa_z \dfrac{\partial D_y}{\partial t} + \frac{\sigma_z}{\epsilon} \partial D_y \label{pre-disc2}
\end{equation}

\begin{equation}
\dfrac{\partial H_y}{\partial x} - \dfrac{\partial H_x}{\partial y} = \kappa_x \dfrac{\partial D_z}{\partial t} + \frac{\sigma_x}{\epsilon} \partial D_z \label{pre-disc3}
\end{equation}

\begin{equation}
 \dfrac{\partial E_z}{\partial y} - \dfrac{\partial E_y}{\partial z} = -\kappa_y \dfrac{\partial B_x}{\partial t} - \frac{\sigma_y}{\epsilon} \partial B_x \label{pre-disc4}
\end{equation}

\begin{equation}
 \dfrac{\partial E_x}{\partial z} - \dfrac{\partial E_z}{\partial x} = -\kappa_z \dfrac{\partial B_y}{\partial t} - \frac{\sigma_z}{\epsilon} \partial B_y \label{pre-disc5}
\end{equation}

\begin{equation}
\dfrac{\partial E_y}{\partial x} - \dfrac{\partial E_x}{\partial y} = -\kappa_x \dfrac{\partial B_z}{\partial t} - \frac{\sigma_x}{\epsilon} \partial B_z \label{pre-disc6}
\end{equation}

\begin{figure}[H]
 
\setlength{\unitlength}{3500sp}%
%
\begin{picture}(6192,5056)(3181,-5583)
\thinlines
{\put(3691,-5101){\framebox(2835,2835){}}
}
{\put(6523,-5097){\line( 3, 2){1433.077}}
}
{\multiput(3688,-5097)(315.50101,210.33401){5}{\line( 3, 2){200.073}}
}
{\put(5131,-1321){\line( 1, 0){2790}}
}
{\put(7921,-1366){\line( 0,-1){2835}}
}
{\multiput(7921,-4156)(-113.87755,0.00000){25}{\line(-1, 0){ 56.939}}
}
{\multiput(5131,-1321)(0.00000,-115.71429){25}{\line( 0,-1){ 57.857}}
}
% {\put(8776,-8026){\vector( 1, 0){585}}
% }
{\put(4861,-5101){\vector( 1, 0){585}}
}
{\put(5041,-2266){\vector( 1, 0){585}}
}
{\put(6076,-1321){\vector( 1, 0){585}}
}
{\put(6523,-2262){\line( 3, 2){1433.077}}
}
{\put(4700,-1591){\vector(-3,-2){405}}
}
{\put(7471,-1636){\vector(-3,-2){405}}
}
{\put(7471,-4471){\vector(-3,-2){405}}
}
{\put(3678,-2267){\line( 3, 2){1433.077}}
}
{\put(3691,-3931){\vector( 0, 1){630}}
}
{\put(6526,-3886){\vector( 0, 1){495}}
}
{\put(7921,-3076){\vector( 0, 1){630}}
}
{\put(5221,-3436){\vector(-1,-1){382.500}}
}
{\put(5851,-1726){\vector( 0, 1){450}}
}
{\put(7246,-3301){\vector( 1, 0){585}}
}
\put(4906,-5416){$H_y$}%
\put(3196,-3526){$H_z$}%
\put(7291,-4921){$H_x$}%
\put(6841,-1636){$H_x$}%
\put(4096,-1546){$H_x$}%
\put(6166,-1186){$H_y$}%
\put(5401,-1591){$E_z$}%
\put(5086,-2086){$H_y$}%
\put(6661,-3796){$H_z$}%
\put(8011,-2896){$H_z$}%
\put(4771,-3436){$E_x$}%
\put(7291,-3076){$E_y$}%

\put(2900,-5416){$(i+\frac{1}{2},j+\frac{1}{2},k-\frac{1}{2})$}%
\end{picture}%
\caption{Célula de Yee}

\end{figure}
Discretizamos as equações de (\ref{pre-disc1}-\ref{pre-disc6}) usando as posições da célula de Yee. Os passos de tem são discretizados usando o algoritmo \textit{leap-frog}:

\begin{eqnarray}
\frac{H_z|_{i+1/2,j+1/2,k}^{n+1/2}-H_z|_{i+1/2,j-1/2,k}^{n+1/2}}{\Delta y}-\frac{H_y|_{i+1/2,j,k+1/2}^{n+1/2}-H_y|_{i+1/2,j,k-1/2}^{n+1/2}}{\Delta z}= \nonumber\\
\frac{\kappa_y}{\Delta t} \left( D_x|_{i+1/2,j,k}^{n+1}-D_x|_{i+1/2,j,k}^{n}\right)+ \frac{\sigma_y}{2 \epsilon} \left(D_x|_{i+1/2,j,k}^{n+1}+D_x|_{i+1/2,j,k}^{n}\right) \,,
\end{eqnarray}

\begin{eqnarray}
\frac{H_x|_{i,j+1/2,k+1/2}^{n+1/2}-H_x|_{i,j+1/2,k-1/2}^{n+1/2}}{\Delta z}-\frac{H_z|_{i+1/2,j+1/2,k}^{n+1/2}-H_z|_{i-1/2,j+1/2,k}^{n+1/2}}{\Delta x}= \nonumber\\
\frac{\kappa_z}{\Delta t} \left( D_y|_{i,j+1/2,k}^{n+1}-D_y|_{i,j+1/2,k}^{n}\right)+ \frac{\sigma_z}{2 \epsilon} \left(D_y|_{i,j+1/2,k}^{n+1}+D_y|_{i,j+1/2,k}^{n}\right) \,,
\end{eqnarray}

\begin{eqnarray}
\frac{H_y|_{i+1/2,j,k+1/2}^{n+1/2}-H_y|_{i-1/2,j,k+1/2}^{n+1/2}}{\Delta x}-\frac{H_x|_{i,j+1/2,k+1/2}^{n+1/2}-H_x|_{i,j-1/2,k+1/2}^{n+1/2}}{\Delta y}= \nonumber\\
\frac{\kappa_x}{\Delta t} \left( D_z|_{i,j,k+1/2}^{n+1}-D_z|_{i,j,k+1/2}^{n}\right)+ \frac{\sigma_x}{2 \epsilon} \left(D_z|_{i,j,k+1/2}^{n+1}+D_z|_{i,j,k+1/2}^{n}\right) \,,
\end{eqnarray}

\begin{eqnarray}
\frac{E_z|_{i,j,k+1/2}^{n+1}-E_z|_{i,j-1,k+1/2}^{n+1}}{\Delta y}-\frac{E_y|_{i,j+1/2,k}^{n+1}-E_y|_{i,j+1/2,k-1}^{n+1}}{\Delta z}= \nonumber\\
-\frac{\kappa_y}{\Delta t} \left( B_x|_{i,j+1/2,k+1/2}^{n+3/2}-B_x|_{i,j+1/2,k+1/2}^{n+1/2}\right)- \frac{\sigma_y}{2 \epsilon} \left(B_x|_{i,j+1/2,k+1/2}^{n+3/2}+B_x|_{i,j+1/2,k+1/2}^{n+1/2}\right) \,,
\end{eqnarray}

\begin{eqnarray}
\frac{E_x|_{i+1/2,j,k}^{n+1}-E_x|_{i+1/2,j,k-1}^{n+1}}{\Delta z}-\frac{E_z|_{i,j,k+1/2}^{n+1}-E_y|_{i-1,j,k+1/2}^{n+1}}{\Delta x}= \nonumber\\
-\frac{\kappa_z}{\Delta t} \left( B_y|_{i+1/2,j,k+1/2}^{n+3/2}-B_y|_{i+1/2,j,k+1/2}^{n+1/2}\right)- \frac{\sigma_z}{2 \epsilon} \left(B_y|_{i+1/2,j,k+1/2}^{n+3/2}+B_y|_{i+1/2,j,k+1/2}^{n+1/2}\right) \,,
\end{eqnarray}

\begin{eqnarray}
\frac{E_y|_{i,j+1/2,k}^{n+1}-E_y|_{i-1,j+1/2,k}^{n+1}}{\Delta x}-\frac{E_x|_{i+1/2,j,k}^{n+1}-E_y|_{i+1/2,j-1,k}^{n+1}}{\Delta y}= \nonumber\\
-\frac{\kappa_x}{\Delta t} \left( B_z|_{i+1/2,j+1/2,k}^{n+3/2}-B_z|_{i+1/2,j+1/2,k}^{n+1/2}\right)- \frac{\sigma_x}{2 \epsilon} \left(B_z|_{i+1/2,j+1/2,k}^{n+3/2}+B_z|_{i+1/2,j+1/2,k}^{n+1/2}\right) \,.
\end{eqnarray}
\\ \\

Isolando adequadamente os temos:
\begin{align}
\alpha_y D_x|_{i+1/2,j,k}^{n+1}=\beta_y D_x|_{i+1/2,j,k}^{n} + \frac{H_z|_{i+1/2,j+1/2,k}^{n+1/2}-H_z|_{i+1/2,j-1/2,k}^{n+1/2}}{\Delta y}-\frac{H_y|_{i+1/2,j,k+1/2}^{n+1/2}-H_y|_{i+1/2,j,k-1/2}^{n+1/2}}{\Delta z}\,,
\end{align}

\begin{align}
\alpha_z D_y|_{i,j+1/2,k}^{n+1} = \beta_z D_y|_{i,j+1/2,k}^{n} + \frac{H_x|_{i,j+1/2,k+1/2}^{n+1/2}-H_x|_{i,j+1/2,k-1/2}^{n+1/2}}{\Delta z}-\frac{H_z|_{i+1/2,j+1/2,k}^{n+1/2}-H_z|_{i-1/2,j+1/2,k}^{n+1/2}}{\Delta x}\,,
\end{align}

\begin{align}
\alpha_x D_z|_{i,j,k+1/2}^{n+1} = \beta_x D_z|_{i,j,k+1/2}^{n} + \frac{H_y|_{i+1/2,j,k+1/2}^{n+1/2}-H_y|_{i-1/2,j,k+1/2}^{n+1/2}}{\Delta x}-\frac{H_x|_{i,j+1/2,k+1/2}^{n+1/2}-H_x|_{i,j-1/2,k+1/2}^{n+1/2}}{\Delta y} \,,
\end{align}

\begin{align}
\alpha_y  B_x|_{i,j+1/2,k+1/2}^{n+3/2}= \beta_y B_x|_{i,j+1/2,k+1/2}^{n+1/2} + \frac{E_y|_{i,j+1/2,k}^{n+1}-E_y|_{i,j+1/2,k-1}^{n+1}}{\Delta z} - \frac{E_z|_{i,j,k+1/2}^{n+1}-E_z|_{i,j-1,k+1/2}^{n+1}}{\Delta y}\,,
\end{align}

\begin{align}
\alpha_z B_y|_{i+1/2,j,k+1/2}^{n+3/2} = \beta_z B_y|_{i+1/2,j,k+1/2}^{n+1/2} + \frac{E_z|_{i,j,k+1/2}^{n+1}-E_z|_{i-1,j,k+1/2}^{n+1}}{\Delta x}-\frac{E_x|_{i+1/2,j,k}^{n+1}-E_x|_{i+1/2,j,k-1}^{n+1}}{\Delta z}\,,
\end{align}

\begin{align}
\alpha_x B_z|_{i+1/2,j+1/2,k}^{n+3/2}= \beta_x B_z|_{i+1/2,j+1/2,k}^{n+1/2} + \frac{E_x|_{i+1/2,j,k}^{n+1}-E_x|_{i+1/2,j-1,k}^{n+1}}{\Delta y} - \frac{E_y|_{i,j+1/2,k}^{n+1}-E_y|_{i-1,j+1/2,k}^{n+1}}{\Delta x}\,,
\end{align}

onde, 

\begin{align}
\alpha_{x,y,z}=\frac{\kappa_{x,y,z}}{\Delta {x,y,z}}+\frac{\sigma_{x,y,z}}{2 \epsilon}\,, 
\end{align}

\begin{align}
\beta_{x,y,z}=\frac{\kappa_{x,y,z}}{\Delta {x,y,z}}+\frac{\sigma_{x,y,z}}{2 \epsilon}\,.
\end{align}

Precisamos também da discretização das relações constitutivas:

\begin{align}
\alpha_z E_x |_{i+1/2,j,k}^{n+1/2} =
\beta_z E_x |_{i+1/2,j,k}^{n-1/2}  +
\frac{1}{\epsilon} \alpha_x  D_x |_{i+1/2,j,k}^{n+1/2}  -
\frac{1}{\epsilon} \beta_x  D_x |_{i+1/2,j,k}^{n-1/2}\,
\end{align}

\begin{align}
\alpha_x E_y |_{i,j+1/2,k}^{n+1/2} =
\beta_x  E_y |_{i,j+1/2,k}^{n-1/2} +
\frac{1}{\epsilon} \alpha_y  D_y |_{i,j+1/2,k}^{n+1/2}  -
\frac{1}{\epsilon} \beta_y  D_y |_{i,j+1/2,k}^{n-1/2}\,
\end{align}

\begin{align}
\alpha_y E_z |_{i,j,k+1/2}^{n+1/2} =
\beta_y E_z |_{i,j,k+1/2}^{n-1/2}  +
\frac{1}{\epsilon} \alpha_z  D_z |_{i,j,k+1/2}^{n+1/2}  -
\frac{1}{\epsilon} \beta_z  D_z |_{i,j,k+1/2}^{n-1/2}\,
\end{align}

\begin{align}
\alpha_z H_x |_{i,j+1/2,k+1/2}^{n+1} = 
\beta_z H_x |_{i,j+1/2,k+1/2}^{n}   +
\frac{1}{\mu} \alpha_x B_x |_{i,j+1/2,k+1/2}^{n+1}  -
\frac{1}{\mu} \beta_x B_x |_{i,j+1/2,k+1/2}^{n}\,.
\end{align}

\begin{align}
\alpha_x H_y |_{i+1/2,j,k+1/2}^{n+1} = 
\beta_x H_y |_{i+1/2,j,k+1/2}^{n} +
\frac{1}{\mu} \alpha_y B_y |_{i+1/2,j,k+1/2}^{n+1}  -
\frac{1}{\mu} \beta_y B_y |_{i+1/2,j,k+1/2}^{n}\,.
\end{align}

\begin{align}
\alpha_y H_z |_{i+1/2,j+1/2,k}^{n+1} = 
\beta_y H_z |_{i+1/2,j+1/2,k}^{n} +
\frac{1}{\mu} \alpha_z B_z |_{i+1/2,j+1/2,k}^{n+1}  -
\frac{1}{\mu} \beta_z B_z |_{i+1/2,j+1/2,k}^{n}\,.
\end{align}


\end{document}
